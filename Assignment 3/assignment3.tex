\documentclass[12pt,letter]{article}
\usepackage{geometry}\geometry{top=0.75in}
\usepackage{amsmath}
\usepackage{amssymb}
\usepackage{mathtools}
\usepackage{xcolor}	% Color words
\usepackage{cancel}	% Crossing parts of equations out
\usepackage{tikz}    	% Drawing 
\usepackage{pgfplots}   % Other plotting
\usepgfplotslibrary{colormaps,fillbetween}
\usepackage{placeins}   % Float barrier
\usepackage{hyperref}   % Links
\usepackage{tikz-qtree} % Trees
\usepackage{graphicx}
\usepackage{subcaption}
\usepackage{array}
\usepackage{enumitem}

%\tikzset{
%    treenode/.style = {shape=rectangle, rounded corners, draw, align=center}
%    root/.style     = {treenode, font=\Large}
%    env/.style      = {treenode, font=\ttyfamily\normalsize},
%    dummy/.style    = {circle,draw}
%}

%\tikzset{every tree node/.style={circle,align=center, anchor=west, grow=right}
%	}
\tikzset{every tree node/.style={align=center,minimum width=2em},%, draw},%, circle},
	 grow=right,
	 level distance=1.75cm}

% Don't indent
\setlength{\parindent}{0pt}
% Function to replace \section with a problem name specifically formatted
\newcommand{\problem}[1]{\vspace{3mm}\Large\textbf{{Problem {#1}\vspace{3mm}}}\normalsize\\}
% Formatting function, like \problem
\newcommand{\ppart}[1]{\vspace{2mm}\large\textbf{\\Part {#1})\vspace{2mm}}\normalsize\\}
% Formatting 
\newcommand{\condition}[1]{\vspace{1mm}\textbf{{#1}:}\normalsize\\}

\newcolumntype{C}{>{\centering\arraybackslash}m{2cm}}

\begin{document}
\title{CIS 572 Assignment 3}
\author{Steven Walton}
\maketitle
\problem{1}
An analyst wants to classify a number of customers based on some given
attributes: total number of accounts, credit utilization (amount of used
credit divided by total credit amount), percentage of on-time payments,
age of credit history, and inquiries (number of time that the customer
requested a new credit account, whether accepted or not). The analyst
acquired some labeled information as shown in the following table:
\begin{table}[h!]
    \centering
    \begin{tabular}{|c|C|c|C|C|c|c|}
        \hline
        ID & Total Accounts & Utilization & Payment History & Age of History (days) & Inquiries & Label\\
        \hline
        1  & 8  & 15\% & 100\% & 1000 & 5 & GOOD\\
        2  & 15 & 19\% & 90\%  & 2500 & 8 & BAD\\
        3  & 10 & 35\% & 100\% & 500  & 8 & BAD\\
        4  & 11 & 40\% & 95\%  & 2000 & 6 & BAD\\
        5  & 12 & 10\% & 99\%  & 3000 & 6 & GOOD\\
        6  & 18 & 15\% & 100\% & 2000 & 5 & GOOD\\
        7  & 3  & 21\% & 100\% & 1500 & 7 & BAD\\
        8  & 14 & 4\%  & 100\% & 3500 & 5 & GOOD\\
        9  & 13 & 5\%  & 100\% & 3000 & 3 & GOOD\\
        10 & 6  & 25\% & 94\%  & 2800 & 9 & BAD\\
        \hline
    \end{tabular}
\end{table}
\FloatBarrier
Consider the following three accounts to be labeled:
\begin{table}[h!]
    \centering
    \begin{tabular}{|C|c|C|C|c|c|}
        \hline
        Total Accounts & Utilization & Payment History & Age of History (days) & Inquiries & Label\\
        \hline
        20 & 50\% & 90\%  & 4500 & 12 & P1\\
        8  & 10\% & 100\% & 550  & 4  & P2\\
        9  & 13\% & 99\%  & 3000 & 6  & P3\\
        \hline
    \end{tabular}
\end{table}
\FloatBarrier
\begin{enumerate}[label=(\alph*)]
    \item Before using nearest neighbor methods to make predictions, how
          would you recommend processing or transforming the data? Why?
          Make any changes you think appropriate to the data before continuing
          on to the next two parts.
    \item What are the predicted labels P1, P2, and P3 using 1-NN with L 1
          distance? Assume that percentages are represented as their corresponding 
          decimal numbers, so 95\% = 0.95. Show your work.
    \item Keep the information of customers 7, 8, 9, and 10 as validation data,
          and find the best K value for the K-NN algorithm. If the best value
          of K is not equal to 1, find the new predictions for P1, P2, and P3.
          Show your work.
\end{enumerate}
\ppart{1}
Before doing nearest neighbor to make predictions I would suggest 
% Incomplete

\ppart{2}



\end{document}
