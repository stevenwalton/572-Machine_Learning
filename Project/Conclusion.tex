\section{Conclusion}
Unfortunately this work did not prove fruitful. While it was not expected to 
show groundbreaking work, we did not expect the results to be as terrible as
they turned out to be. From these results we can conclude that a UNet is
not the appropriate model to use for generating anything close to 
streamline images nor for identifying regions of interest. We believe that
a substantially more complicated network will need to be used to make
progress.

\subsection{Future Work}
Being an open problem and that machine learning in scientific visualization
is a relatively unexplored work there is more than ample reason to continue
to pursue this problem. While replicating the model from Han et al was outside
the scope of this class, it seems like a reasonable next step. Verifying and
testing their model would be the best stepping stone to work from. This 
would be good work to attempt over the summer. It would
be interesting to test variational autoencoders and test against the results
of FlowNet. This is suspected to be a significant undertaking and that tuning
the hyper-parameters will take significant time. Work also needs to be done 
to generalize a machine learning model to generate streamlines. This is the
major downfalls of the works referenced in the Background section. For a 
machine learning technique to truly be considered successful it needs to be
able to generalize its solution. While the referenced work was insightful
and shows promise, there is much work to be done before we see machine
learning used in the generation of streamlines.

While Han et al. and Li et al.'s work focused on similarity based
techniques there are, to the author's knowledge, no research done in exploring
machine learning through density or feature based streamline generation. It
also appears that work has not been done to directly take in vector field data
and generate streamlines from this. Since one can think of streamlines as 
an encoding of vector fields it may be interesting to explore the use
of various types of encoders, in combination with other networks, to generate
streamlines. Specifically one can think about features being the key features
that need to be encoded, so it may be possible to be extract features from 
vector field data. Potentially some form of CNN may be able to extract these
features as well. We believe that the promising direction for generating 
streamlines from vector field data would be using a feature based technique,
where a model could identify the key features previously listed. If a model 
only showed these features this would be considered major progress and show 
promise for this type of work.
