\begin{abstract}
Generating streamlines are an integral part of flow visualization. They 
enable scientists and engineers to accurately determine how fluids are
acting and allow decisions to be made based off of these results. Generating
streamlines that show interesting regions, are uniformly distributed, 
aesthetically pleasing, and do not excessively computationally expensive
remains an open problem in scientific visualization. This work attempts
to bring light to how machine learning can be used to tackle this problem. A
UNet is used to perform image segmentation and ultimately results show that 
this is not a promising avenue of study. Other models may be more fruitful.
\end{abstract}
