\Section{Background}
\subsection{Streamlines}
A streamline is defined as a line that shows how fluid flows. These are used in 
scientific visualization to help scientists visualize what is happening within
fluids. Generally these are used because the vector field, which is what is used
in the simulation data, is too information dense and difficult to make sense of.
A good streamline is one that has sparse representation of the fluid
but also shows all key features. 

This is an unsolved problem in scientific visualization because it is hard to 
define what a good representation is. Key features are also difficult to define.
Features that are important are sources, sinks, and flow direction. What can
be difficult is the interaction between these features. Straight flow lines 
bend around sources and sinks even if they do not enter them. This creates a lot
of open questions. Such as: ``How different do streamlines need to be?" and 
``At what point do we start drawing and terminate a line?". 
\end{section}
