\section{Background}
Scientists use flow visualization techniques to understand the basics of how
fluids move in various conditions. This can range from the water flowing in
the oceans, global wind data, to how fuel moves inside a rocket engine. 
Understanding fluid flow has been a great challenge for scientists for a long
time. 

When solving these systems scientists use vector fields placed at each location
within the geometry, denoting the direction and magnitude of the fluid flow. 
By calculating these vectors a scientist can track how a particle moves about
its environment. Computation becomes increasingly difficult when these vector 
fields start changing as a function of time, these simulations are more 
representative of real world phenomena. 

The difficulty with vector field representations is that it would be impossible 
to visualize. If we have to draw a vector, with length representing magnitude, 
at every single point in a geometry then they would completely overlap and 
you'd just see a completely black image. This can be partially solved by using
a sparse representation of vectors. Figure~\ref{fig:vec} shows an example
of this type of sparse representation. In this figure we can see that there are 
two sinks, areas where flow goes into (such as a kitchen sink), and many 
areas where fluid flows at different velocities and directions. There is a lot
of information to process in this image and it can be difficult to tell what
is happening.

\begin{figure}[ht]
    \centering
    \includegraphics[width=0.5\textwidth]{vecField.jpg}
    \caption{Sample vector field}
    \label{fig:vec}
\end{figure}


\FloatBarrier
\subsection{Streamlines}
A streamline is defined as a line that shows how fluid flows. These are used in 
scientific visualization to help scientists visualize what is happening within
fluids. Generally these are used because the vector field, which is what is used
in the simulation data, is too information dense and difficult to make sense of.
A good streamline is one that has sparse representation of the fluid
but also shows all key features. Figure~\ref{fig:teapot} shows an example of 
streamlines in a teapot, representing the heat. We can see that 
while this representation is easier to understand than those from the vector 
field, it is still cumbersome. 
\begin{figure}[ht]
    \centering
    \includegraphics[width=0.4\textwidth]{potstreamlines.png}
    \caption{Streamlines Depicting Heat in a Tea Pot}
    \label{fig:teapot}
\end{figure}

\begin{figure}[ht]
    \centering
    \includegraphics[width=0.8\textwidth]{streamlines.png}
    \caption{Different Streamline Representations of the Same Fluid Flow with
    50 Streamlines\cite{Sane}}
    \label{fig:sl}
\end{figure}

Generating streamlines that are both sparse and representative is an unsolved 
problem in scientific visualization because it is hard to 
define what a good representation is. Key features are also difficult to define.
Features that are important are sources, sinks, and flow direction. What can
be difficult is the interaction between these features. Straight flow lines 
bend around sources and sinks even if they do not enter them. This creates a lot
of open questions. Such as: ``How different do streamlines need to be?" and 
``At what point do we start drawing and terminate a line?". 
Verma eta al.\cite{Verma} were the first to define the desired characteristics
of streamlines, naming the three criteria as: Coverage; Uniformity; and 
Continuity. Coverage refers to ensuring that streamlines convey key features,
or regions of interest. Uniformity is defined such that streamlines should
be uniformly distributed over a field, meaning that there shouldn't be large 
empty spaces in some areas and highly dense areas in others. Continuity is
defined as an aesthetic interest, mainly in that longer streamlines are preferred.
Figure~\ref{fig:sl}
shows how two different streamline representations can convey different 
information. The red areas in Image B show different sources and sinks that
aren't as apparent or sometimes non-existent. In the blue regions of Image
A we can see some repetitive lines, ones that aren't in Image B. 

\FloatBarrier
\subsection{Techniques to Generate Streamlines}
Sane~\cite{Sane} collected a large number of modern techniques used for 
streamline selection for his area exam. Within this work he discusses the use
of seed placement and streamline selections. That is by placing a point within
a vector field, following its movement and selecting proper streamlines to draw
based on these seed placements. In this paper he lists the common practices
for generating streamlines. The three major categories are: density based
techniques, feature based, and similarity based techniques. Density based
focuses on creating evenly-spaced streamlines within the image space. Feature
based techniques are formed from vector fields and focus on identifying key
features, such as: centers, sources, sinks, repelling-spirals, 
attracting-spirals, and saddles. See Figure~\ref{fig:features}.

\begin{figure}[ht]
    \centering
    \includegraphics[width=0.7\textwidth]{Features.png}
    \caption{Types of Features~\cite{Verma}}
    \label{fig:features}
\end{figure}

\FloatBarrier
Similarity based techniques focus on reducing the number of streamlines to 
reduce the similarity between different lines. Figure~\ref{fig:sim} shows 
three different streamlines that have different types of similarities. It
is clear that in (a) that these lines are redundant because $p_i$ is close
to $q_i$. Conversely we can see that while (b) and (c) look similar they
go in different directions. For similarity based techniques we not only care
about the shape of the streamlines, but also the orientation similarity.

\begin{figure}[ht]
    \centering
    \includegraphics[width=0.7\textwidth]{Similarity.png}
    \caption{Showing Similarity in (a) and differences in (b) and (c)~\cite{Chen}}
    \label{fig:sim}
\end{figure}
\FloatBarrier

In the similarity based techniques approach we find a subsection of machine
learning that includes work with SVM's~\cite{Li} and DNN Feature Descriptions
\cite{Han}. Li's~\cite{Li} work focuses on uses binary support vector machines
to determine which streamlines are similar and removes ones that are considered
too similar. This method work does not account for what humans consider to be
important. Han et al.'s~\cite{Han} work uses a CNN across 3D image data and then
uses an autoencoder to learn what the minimum number of streamlines are needed
to represent a cluster of streamlines. They use the sum of the Euclidean distance
to all other points in a cluster and identify its minimum. While this technique
was successful it takes a significant amount of time to train and does not 
generalize to arbitrary streamlines. Figure~\ref{fig:FN} shows the machine
learning model used by Han et al. The voxel information is the 3D image data
that is being used as an input into the 3D convolutional layer. The inability
to create a generalized model is the major drawback of this technique.

\begin{figure}[ht]
    \centering
    \includegraphics[width=1.1\textwidth]{FlowNet.png}
    \caption{FlowNet Model}
    \label{fig:FN}
\end{figure}
