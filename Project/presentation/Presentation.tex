\documentclass[pdf,11pt]{beamer}

\usepackage[utf8]{inputenc}
\usepackage{graphicx}       % Images
\graphicspath{{Images/}}
\usepackage{xcolor}         % Change Colors
\usepackage{caption}
\captionsetup[figure]{font=small}
\usepackage{multimedia}     % Movies!
\usepackage{tikz}           % Vectored pictures
%\usepackage{media9}
%\hypersetup{pdfpagemode=FullScreen} % Presentation Mode

% Title
\title{Generating Streamlines with Machine Learning}
\author{Steven Walton\\ \small University of Oregon}
\date{CIS 572: Machine Learning\\14 June 2019}

% Background
\usebackgroundtemplate
{
    \includegraphics[width=\paperwidth,height=\paperheight]{UOTemplate.png}
}
\definecolor{UOYellow}{HTML}{FDCB00}
%\newcommand{\mytitle}[1]{\setcolor{bg=UOYellow,fg=green}\frametitle{{#1}}}

% Formatting
\setbeamercolor{title}{fg=white}
\setbeamercolor{normal text}{fg=white}
\setbeamercolor{structure}{fg=white}        % Adjusts figures and frame titles
\setbeamertemplate{frametitle}[default][center]
\setbeamerfont{frametitle}{size=\Huge}
%\setbeamerfont{normal text}{size=\LARGE}
%\setbeamerfont{structure}{size=\LARGE}

% Some personal options for trees
\tikzset{
    ->, 
    level distance = 12em,
    minimum size=2em,
    %edge from parent/.style={draw,thick},
    level 1/.style={sibling distance=6em},
    level 2/.style={sibling distance=3em},
    thick/.style = {line width=1.5pt},
    extra thick/.style = {line width=3.5pt},
    red node/.style={shape=circle,draw=red,fill=red!40,thick,inner sep=1.2},
    blue node/.style={shape=circle,draw=blue,fill=blue!40,thick,inner sep=1.2}
}


% Make quadrant minipage
\newcommand\Quadrents[4]{
    \begin{minipage}[b][0.35\textheight][t]{0.49\textwidth}#1\end{minipage}\hfill
    \begin{minipage}[b][0.35\textheight][t]{0.49\textwidth}#2\end{minipage}\\[0.5em]
    \begin{minipage}[b][0.35\textheight][t]{0.49\textwidth}#3\end{minipage}\hfill
    \begin{minipage}[b][0.35\textheight][t]{0.49\textwidth}#4\end{minipage}
}

\newcommand\Split[2]{
    \begin{minipage}[b][0.35\textheight][t]{0.8\textwidth}#1\end{minipage}\\[2.0em]
    \begin{minipage}[b][0.35\textheight][t]{0.8\textwidth}#2\end{minipage}
}

\begin{document}
\frame{\titlepage}
\begin{frame}
\frametitle{What Are Streamlines?}
\begin{columns}
    \begin{column}{0.49\paperwidth}
        \begin{itemize}
            \item Field lines that show fluid flow
            \item Tell scientists and engineers how fluids move around objects
            \item Simulations create vector fields which we need to translate
        \end{itemize}
    \end{column}

    \begin{column}{0.49\paperwidth}
    \begin{figure}
        \centering
        \includegraphics[height=0.5\textwidth]{potstreamlines.png}
        \caption{Dense streamline of heated teapot}
    \end{figure}
    \end{column}
\end{columns}
\end{frame}

\begin{frame}
\frametitle{What is a "good" streamline representation?}
    \begin{columns}
        \begin{column}{0.49\paperwidth}
        \begin{itemize}
            \item An unsolved problem!
            \item Want something that shows key features in fluid flow
            \item Don't want repetitive lines 
            \item We want a compact and concise representation.
        \end{itemize}
        \end{column}
%
\begin{column}{0.49\paperwidth}
    \begin{figure}
        \centering
        \includegraphics[height=0.5\textwidth]{Harpoon.jpg}
        \caption{Streamlines of Harpoon missile under water}
    \end{figure}
\end{column}
\end{columns}
\end{frame}

\begin{frame}
    \frametitle{Let's Ignore What is Good and Prototype}
    \begin{columns}
        \begin{column}{0.49\paperwidth}
            \begin{itemize}
                \item We'll concentrate on seeing if we can reduce complexity
                \item We'll test supervised learning to see if we can encode basic info
            \end{itemize}
        \end{column}

        \begin{column}{0.49\paperwidth}
            \begin{figure}
                \centering
                \includegraphics[width=0.4\textwidth]{ABC_In.png}
            \end{figure}
            \begin{figure}
                \centering
                \includegraphics[width=0.4\textwidth]{ABC_Out.png}
            \end{figure}

        \end{column}
    \end{columns}
\end{frame}

\begin{frame}
    \frametitle{Let's Try Image Segmentation}
    \begin{columns}
        \begin{column}{0.49\paperwidth}
            \begin{itemize}
                \item Probably want some type of encoder
                \item UNets are great at performing image segmentation
                \item Don't expect to fully work, but let's see if we're on the right track
            \end{itemize}
        \end{column}

        \begin{column}{0.49\paperwidth}
            \begin{figure}
                \centering
                \includegraphics[width=0.8\textwidth]{unet.jpg}
                \caption{Sample Unet Configuration}
            \end{figure}
        \end{column}
    \end{columns}
\end{frame}

\begin{frame}
    \frametitle{Let's Train!}
    \includegraphics<2>[width=0.8\textwidth]{out.png}
    \includegraphics<3>[width=0.8\textheight]{SGD_Loss_lr0p01_batch20_mom0p5.png}
    \includegraphics<4>[width=0.8\textheight]{SGD_Loss_lr0p01_batch20_mom0p8.png}
    \includegraphics<5>[width=0.8\textheight]{SGD_Loss_lr0p01_batch20_mom0p9.png}
    \includegraphics<6>[width=0.8\textheight]{SGD_Loss_lr0p05_batch20_mom0p5.png}
    \includegraphics<7>[width=0.8\textheight]{SGD_Loss_lr0p05_batch20_mom0p8.png}
    \includegraphics<8>[width=0.8\textheight]{SGD_Loss_lr0p05_batch20_mom0p9.png}
    \includegraphics<9>[width=0.8\textheight]{SGD_Loss_lr0p1_batch20_mom0p5.png}
    \includegraphics<10>[width=0.8\textheight]{SGD_Loss_lr0p1_batch20_mom0p8.png}
    \includegraphics<11>[width=0.8\textheight]{SGD_Loss_lr0p1_batch20_mom0p9.png}
    \includegraphics<12>[width=0.8\textheight]{SGD_Loss_lr0p2_batch20_mom0p5.png}
    \includegraphics<13>[width=0.8\textheight]{SGD_Loss_lr0p2_batch20_mom0p8.png}
    \includegraphics<14>[width=0.8\textheight]{SGD_Loss_lr0p2_batch20_mom0p9.png}
    \includegraphics<15>[width=0.8\textheight]{SGD_Loss_lr0p5_batch20_mom0p5.png}
    \includegraphics<16>[width=0.8\textheight]{SGD_Loss_lr0p5_batch20_mom0p8.png}
    \includegraphics<17>[width=0.8\textheight]{SGD_Loss_lr0p5_batch20_mom0p9.png}
    \includegraphics<18>[width=0.8\textheight]{SGD_Loss_lr0p8_batch20_mom0p5.png}
    \includegraphics<19>[width=0.8\textheight]{SGD_Loss_lr0p8_batch20_mom0p8.png}
    \includegraphics<20>[width=0.8\textheight]{SGD_Loss_lr0p8_batch20_mom0p9.png}
    \includegraphics<21>[width=0.8\textheight]{ADAM_Loss_lr0p01_batch20.png}
    \includegraphics<22>[width=0.8\textheight]{ADAM_Loss_lr0p05_batch20.png}
    \includegraphics<23>[width=0.8\textheight]{ADAM_Loss_lr0p1_batch20.png}
    \includegraphics<24>[width=0.8\textheight]{ADAM_Loss_lr0p2_batch20.png}
    \includegraphics<25>[width=0.8\textheight]{ADAM_Loss_lr0p5_batch20.png}
    \includegraphics<26>[width=0.8\textheight]{ADAM_Loss_lr0p8_batch20.png}
\end{frame}

\begin{frame}
    \frametitle{Lessons To Learn}
    \begin{itemize}
        \item<1-> Research is HARD!
        \item<2-> Just because it is ML doesn't mean it can learn everything.
        \item<3-> Searching around shows that autoencoders might be promising.
        \item<4-> Every failure is another step in the process of solving hard problems.
    \end{itemize}
\end{frame}

%\begin{frame}
%    \frametitle{Let's Ignore What is Good and Prototype}
%    \begin{columns}
%        \begin{column}{0.49\paperwidth}
%        \end{column}
%
%
%        \begin{column}{0.49\paperwidth}
%        \end{column}
%    \end{columns}
%\end{frame}

\begin{frame}
\centering
\Huge{Questions?}\\
\normalsize{(Let's look at code)}
\end{frame}
\end{document}

